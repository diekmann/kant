\documentclass[11pt,a4paper]{article}
\usepackage[T1]{fontenc}
\usepackage[utf8]{inputenc}
\usepackage{isabelle,isabellesym}


\usepackage{marvosym} % world symbol
\newcommand{\isactrlurl}[0]{\Mundus}

\usepackage{graphicx}

% further packages required for unusual symbols (see also
% isabellesym.sty), use only when needed

%\usepackage{amssymb}
  %for \<leadsto>, \<box>, \<diamond>, \<sqsupset>, \<mho>, \<Join>,
  %\<lhd>, \<lesssim>, \<greatersim>, \<lessapprox>, \<greaterapprox>,
  %\<triangleq>, \<yen>, \<lozenge>

\usepackage{eurosym}
  %for \<euro>

%\usepackage[only,bigsqcap,bigparallel,fatsemi,interleave,sslash]{stmaryrd}
  %for \<Sqinter>, \<Parallel>, \<Zsemi>, \<Parallel>, \<sslash>

%\usepackage{eufrak}
  %for \<AA> ... \<ZZ>, \<aa> ... \<zz> (also included in amssymb)

%\usepackage{textcomp}
  %for \<onequarter>, \<onehalf>, \<threequarters>, \<degree>, \<cent>,
  %\<currency>

\usepackage{tikz}
\usepackage{amsmath,amssymb,tikz-cd}

\columnsep 2pc
\textwidth 40pc
\oddsidemargin 4.5pc
\evensidemargin 4.5pc
\advance\oddsidemargin by -1.11in  % Correct for LaTeX gratuitousness
\advance\evensidemargin by -1.11in % Correct for LaTeX gratuitousness
\marginparwidth 0pt             % Margin pars are not allowed.
\marginparsep 11pt              % Horizontal space between outer margin and
\emergencystretch=10cm


% this should be the last package used
\usepackage{pdfsetup}

% urls in roman style, theory text in math-similar italics
\urlstyle{rm}
\isabellestyle{it}

% for uniform font size
%\renewcommand{\isastyle}{\isastyleminor}

\isadroptag{theory}

\begin{document}

\title{Einführung in Mathematische Modellierung mit Isabelle/HOL am Beispiel Philosophie:\\
Extensionale Interpretation des Kategorischen Imperativs}
\author{Cornelius Diekmann}
\maketitle

\begin{abstract}

Language: German.

\medskip

In diesem Artikel modellieren wir eine persönlich angehauchte
Interpretation von Kants kategorischem Imperativ.
Primär ist dieses Dokument eine Einführung in die Modellierung mit Isabelle/HOL
am Beispiel Philosophie.
Das bevorzugte Leseformat ist die Proof Document Outline,
welche alle wichtigen Definitionen, Beispiele und Ergbnisse (Lemmata und Theoreme) beinhaltet,
jedoch die eigentlichen Beweise auslässt.
Dieses Dokument ist im Theorembeweiser Isabelle/HOL geschrieben,
d.h.\ dass beim Bau des Dokuments alle Beweise vom Computer überprüft werden.
Wir können uns daher sicher sein, dass die Beweise stimmen
und sie guten Gewissens im finalen Dokument auslassen.
Dies erlaubt es uns, uns ganz auf die Feinheiten der Definitionen
und Bedeutung der Behauptungen zu fokussieren.


Bei der "extensionalen Interpretation des Kategorischen Imperativs" handelt es sich
um meine persönliche Auslegung von Kants kategorischem Imperativ,
welche teilweise Kant widerspricht.
Der Hautpunterschied ist die Extensionalität, wie später im Dokument erläutert wird.
Dies erlaubt es uns, ein \emph{shallow embedding} morlaischer Definitionen in
unsere mathematische Logik zu konstruieren.

Meine primäre Inspiration ist die Sekundärliteratur von Bertrand Russell.
Russel war sowohl klassischer Mathematiker, einer der Urherber der Principia Mathematica,
als auch Philosoph.
Ich behaupte, hätte Russel damals Zugang zu Isabelle/HOL gehabt, wären wir heute weiter
und Higher-Order Logik (HOL) wäre Russels Logik der Wahl geworden.

Tiefgreifende neue philosophische Einsichten werden wir in diesem Artikel nicht entwicken.
In Beispielen werden werden wir einige bereits bekannte Ergebnisse sehen:
Z.B.\ Steheln ist schlecht, etwas Wertvolles erschaffen ist gut,
Eigentumsübergang ist nur okay wenn Konsens herrscht,
bei der Bemessung von Steuern sollte jeder gleich behandelt werden.
Diese Einsichten sind weder neu noch überraschend.
Der interessante Punkt ist jedoch,
dass wir diese Einsichten aus der sehr generischen, allgemeinen und selbstreferentielle Formalisierung
unseres kategorischen Imperativs herleiten werden!
Während unser Artikel mit diesen Beispielen abschließen wird,
steht dem allgemeinen Konzept und der Möglichkeit komplexere
und umfassendere Weltenmodelle zu entwickeln um kompliziertere Fragestellungen zu diskutieren
nichts im Weg.
\end{abstract}

% Refernce:
% Lawrence Paulson @LawrPaulson Replying to @popitter_net and @analytichegel
% "Yes — PM goes easily into higher-order logic"
% where "PM" is Principia Mathematicafrom context.
%
%
%https://twitter.com/LawrPaulson/status/1582788684050358272

\tableofcontents

% sane default for proof documents
\parindent 0pt\parskip 0.5ex

% generated text of all theories
\input{session}

% optional bibliography
\bibliographystyle{abbrv}
\bibliography{root}

\end{document}

%%% Local Variables:
%%% mode: latex
%%% TeX-master: t
%%% End:
